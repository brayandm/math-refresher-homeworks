\textbf{Problem 1:}

\singlespacing

In this problem we have an unfair coin with probability of heads $p$
and probability of tails $1-p$.

\singlespacing

To make a fair coin or a fair toss we can throw the unfair coin twice,
so we have 4 possible outcomes:

\singlespacing

\begin{enumerate}
    \item \textbf{HH} with probability equals to $p^2$
    \item \textbf{HT} with probability equals to $p(1-p)$
    \item \textbf{TH} with probability equals to $(1-p)p$
    \item \textbf{TT} with probability equals to $(1-p)^2$
\end{enumerate}

\singlespacing

And then because we have two outcomes with same probability \textbf{HT}
and \textbf{TH}, then we can assign heads to \textbf{HT} and tails to
\textbf{TH} and we have a fair coin. But there is a case when we get
\textbf{HH} or \textbf{TT}, in that case we need to repeat the process
until we get \textbf{HT} or \textbf{TH}.

\singlespacing

Let's define the toss of this two coins as a step, so our problem is to calculate
the expected number of steps to get \textbf{HT} or \textbf{TH}.

\singlespacing

We know the probability of getting \textbf{HT} or \textbf{TH} in one step, that
is $p(1-p) + (1-p)p = 2p(1-p)$. And we know that the probability of getting
\textbf{HH} or \textbf{TT} in one step is $p^2 + (1-p)^2$.

\singlespacing

In each step we have a probability of $2p(1-p)$ of getting \textbf{HT} or
\textbf{TH} and then we finish the process, and a probability of $p^2 + (1-p)^2$
of getting \textbf{HH} or \textbf{TT} and then we need to repeat the same process.
So defining $X$ as the number of steps to get \textbf{HT} or \textbf{TH}, we have:

\singlespacing

\begin{equation}
    E[X] = 1 + 2p(1-p) \cdot 0 + (p^2 + (1-p)^2)\cdot E[X]
\end{equation}

\singlespacing

Let's analyze the equation above, we have add $1$ because we always do the first step,
then we have $2p(1-p) \cdot 0$ because if we get \textbf{HT} or \textbf{TH} in the first
step we don't need to repeat the process anymore so we have $0$ steps, and then we have
$(p^2 + (1-p)^2)\cdot E[X]$ because if we get \textbf{HH} or \textbf{TT} in the first
step we need to repeat the process (because the process start again and it is the same
problem, we just need to multiply the expected number of steps by the probability of
getting \textbf{HH} or \textbf{TT} in one step, so the expected number of steps
depends on itself).

\singlespacing

Solving the equation above we have:

\singlespacing

\begin{equation}
    E[X] = 1 + 2p(1-p) \cdot 0 + (p^2 + (1-p)^2)\cdot E[X]
\end{equation}

\singlespacing

\begin{equation}
    E[X] = 1 + (p^2 + (1-p)^2)\cdot E[X]
\end{equation}

\singlespacing

\begin{equation}
    E[X] - (p^2 + (1-p)^2)\cdot E[X] = 1
\end{equation}

\singlespacing

\begin{equation}
    E[X] \cdot (1 - (p^2 + (1-p)^2)) = 1
\end{equation}

\singlespacing

\begin{equation}
    E[X] = \frac{1}{1 - (p^2 + (1-p)^2)}
\end{equation}