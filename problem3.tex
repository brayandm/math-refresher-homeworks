\textbf{Problem 3:}

\singlespacing

$\textbf{a)}$

\singlespacing

\begin{math}
    A=\begin{pmatrix}
        \begin{array}{ccc}
            2 & 3 & 4 \\
            3 & 4 & 5 \\
            4 & 5 & 6
        \end{array}
    \end{pmatrix}
\end{math}

\singlespacing

\begin{math}
    det(A) = 2*4*6 + 3*5*4 + 4*3*5 - 4*4*4 - 3*3*6 - 2*5*5 = 0
\end{math}

\singlespacing

So the matrix is singular and it is not invertible.

\singlespacing

\singlespacing

$\textbf{b)}$

\singlespacing

\begin{math}
    A = \begin{pmatrix}
        \begin{array}{cccc}
            1 & 0 & 1 & 0 \\
            0 & 1 & 1 & 0 \\
            1 & 1 & 0 & 1 \\
            1 & 1 & 1 & 0
        \end{array}
    \end{pmatrix}
\end{math}

\singlespacing

Aplying the Gauss-Jordan elimination method:

\singlespacing
\singlespacing

\begin{math}
    \begin{pmatrix}
        \begin{array}{cccc|cccc}
            1 & 0 & 1 & 0 & 1 & 0 & 0 & 0 \\
            0 & 1 & 1 & 0 & 0 & 1 & 0 & 0 \\
            1 & 1 & 0 & 1 & 0 & 0 & 1 & 0 \\
            1 & 1 & 1 & 0 & 0 & 0 & 0 & 1
        \end{array}
    \end{pmatrix}
\end{math}

\singlespacing

\begin{math}
    \begin{pmatrix}
        \begin{array}{cccc|cccc}
            1 & 0 & 1  & 0 & 1  & 0 & 0 & 0 \\
            0 & 1 & 1  & 0 & 0  & 1 & 0 & 0 \\
            0 & 1 & -1 & 1 & -1 & 0 & 1 & 0 \\
            1 & 1 & 1  & 0 & 0  & 0 & 0 & 1
        \end{array}
    \end{pmatrix}
    r3 = r3  + (-r1)
\end{math}

\singlespacing

\begin{math}
    \begin{pmatrix}
        \begin{array}{cccc|cccc}
            1 & 0 & 1  & 0 & 1  & 0 & 0 & 0 \\
            0 & 1 & 1  & 0 & 0  & 1 & 0 & 0 \\
            0 & 1 & -1 & 1 & -1 & 0 & 1 & 0 \\
            0 & 1 & 0  & 0 & -1 & 0 & 0 & 1
        \end{array}
    \end{pmatrix}
    r4 = r4  + (-r1)
\end{math}

\singlespacing

\begin{math}
    \begin{pmatrix}
        \begin{array}{cccc|cccc}
            1 & 0 & 1  & 0 & 1  & 0  & 0 & 0 \\
            0 & 1 & 1  & 0 & 0  & 1  & 0 & 0 \\
            0 & 0 & -2 & 1 & -1 & -1 & 1 & 0 \\
            0 & 1 & 0  & 0 & -1 & 0  & 0 & 1
        \end{array}
    \end{pmatrix}
    r3 = r3  + (-r2)
\end{math}


\singlespacing

\begin{math}
    \begin{pmatrix}
        \begin{array}{cccc|cccc}
            1 & 0 & 1  & 0 & 1  & 0  & 0 & 0 \\
            0 & 1 & 1  & 0 & 0  & 1  & 0 & 0 \\
            0 & 0 & -2 & 1 & -1 & -1 & 1 & 0 \\
            0 & 0 & -1 & 0 & -1 & -1 & 0 & 1
        \end{array}
    \end{pmatrix}
    r4 = r4  + (-r2)
\end{math}

\singlespacing

\begin{math}
    \begin{pmatrix}
        \begin{array}{cccc|cccc}
            1 & 0 & 1  & 0    & 1   & 0   & 0    & 0 \\
            0 & 1 & 1  & 0    & 0   & 1   & 0    & 0 \\
            0 & 0 & 1  & -1/2 & 1/2 & 1/2 & -1/2 & 0 \\
            0 & 0 & -1 & 0    & -1  & -1  & 0    & 1
        \end{array}
    \end{pmatrix}
    r3 = r3 / (-2)
\end{math}

\singlespacing

\begin{math}
    \begin{pmatrix}
        \begin{array}{cccc|cccc}
            1 & 0 & 0  & 1/2  & 1/2 & -1/2 & 1/2  & 0 \\
            0 & 1 & 1  & 0    & 0   & 1    & 0    & 0 \\
            0 & 0 & 1  & -1/2 & 1/2 & 1/2  & -1/2 & 0 \\
            0 & 0 & -1 & 0    & -1  & -1   & 0    & 1
        \end{array}
    \end{pmatrix}
    r1 = r1 + (-f3)
\end{math}

\singlespacing

\begin{math}
    \begin{pmatrix}
        \begin{array}{cccc|cccc}
            1 & 0 & 0  & 1/2  & 1/2  & -1/2 & 1/2  & 0 \\
            0 & 1 & 0  & 1/2  & -1/2 & 1/2  & 1/2  & 0 \\
            0 & 0 & 1  & -1/2 & 1/2  & 1/2  & -1/2 & 0 \\
            0 & 0 & -1 & 0    & -1   & -1   & 0    & 1
        \end{array}
    \end{pmatrix}
    r2 = r3 + (-f3)
\end{math}

\singlespacing

\begin{math}
    \begin{pmatrix}
        \begin{array}{cccc|cccc}
            1 & 0 & 0 & 1/2  & 1/2  & -1/2 & 1/2  & 0 \\
            0 & 1 & 0 & 1/2  & -1/2 & 1/2  & 1/2  & 0 \\
            0 & 0 & 1 & -1/2 & 1/2  & 1/2  & -1/2 & 0 \\
            0 & 0 & 0 & -1/2 & -1/2 & -1/2 & -1/2 & 1
        \end{array}
    \end{pmatrix}
    r4 = r4 + f3
\end{math}

\singlespacing

\begin{math}
    \begin{pmatrix}
        \begin{array}{cccc|cccc}
            1 & 0 & 0 & 1/2  & 1/2  & -1/2 & 1/2  & 0  \\
            0 & 1 & 0 & 1/2  & -1/2 & 1/2  & 1/2  & 0  \\
            0 & 0 & 1 & -1/2 & 1/2  & 1/2  & -1/2 & 0  \\
            0 & 0 & 0 & 1    & 1    & 1    & 1    & -2
        \end{array}
    \end{pmatrix}
    r4 = r4 * (-2)
\end{math}

\singlespacing

\begin{math}
    \begin{pmatrix}
        \begin{array}{cccc|cccc}
            1 & 0 & 0 & 1/2 & 1/2  & -1/2 & 1/2 & 0  \\
            0 & 1 & 0 & 1/2 & -1/2 & 1/2  & 1/2 & 0  \\
            0 & 0 & 1 & 0   & 1    & 1    & 0   & -1 \\
            0 & 0 & 0 & 1   & 1    & 1    & 1   & -2
        \end{array}
    \end{pmatrix}
    r3 = r3 + (r4/2)
\end{math}

\singlespacing

\begin{math}
    \begin{pmatrix}
        \begin{array}{cccc|cccc}
            1 & 0 & 0 & 1/2 & 1/2 & -1/2 & 1/2 & 0  \\
            0 & 1 & 0 & 0   & -1  & 0    & 0   & 1  \\
            0 & 0 & 1 & 0   & 1   & 1    & 0   & -1 \\
            0 & 0 & 0 & 1   & 1   & 1    & 1   & -2
        \end{array}
    \end{pmatrix}
    r2 = r2 + (-r4/2)
\end{math}

\singlespacing

\begin{math}
    \begin{pmatrix}
        \begin{array}{cccc|cccc}
            1 & 0 & 0 & 0 & 0  & -1 & 0 & 1  \\
            0 & 1 & 0 & 0 & -1 & 0  & 0 & 1  \\
            0 & 0 & 1 & 0 & 1  & 1  & 0 & -1 \\
            0 & 0 & 0 & 1 & 1  & 1  & 1 & -2
        \end{array}
    \end{pmatrix}
    r1 = r1 + (-r4/2)
\end{math}

\singlespacing

So the $A^{-1} = \begin{pmatrix}
        \begin{array}{cccc}
            0  & -1 & 0 & 1  \\
            -1 & 0  & 0 & 1  \\
            1  & 1  & 0 & -1 \\
            1  & 1  & 1 & -2
        \end{array}
    \end{pmatrix}$