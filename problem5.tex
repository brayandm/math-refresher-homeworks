\textbf{Problem 5:}

\singlespacing

\begin{math}
    A_1=\begin{bmatrix}
        \begin{array}{ccc}
            1 & 0  & 1  \\
            1 & -2 & -1 \\
            2 & 1  & 3  \\
            1 & 0  & 1
        \end{array}
    \end{bmatrix},\quad
    A_2=\begin{bmatrix}
        \begin{array}{ccc}
            3 & -3 & 0 \\
            1 & 2  & 3 \\
            7 & -5 & 2 \\
            3 & -1 & 2
        \end{array}
    \end{bmatrix}
\end{math}

\singlespacing
\singlespacing

Let's find the basis of $U_1$ and $U_2$ where where $U_1$ is spanned by the columns
of $A_1$ and $U_2$ is spanned by the columns of $A_2$.

\singlespacing

We can notice that in $A_1$ the sum of the first and second column is equal to the
third column, so we can say that the third column is a linear combination of the
first and second column. Then for now we can generate $U_1$ with the first and
second column of $A_1$.

\singlespacing

But because the first and second column are linearly independent, we can say that
the basis of $U_1$ is the first and second column of $A_1$.

\singlespacing

The proof that the first and second column of $A_1$ are linearly independent is
that the only solution to the equation:

\singlespacing

\begin{math}
    a \begin{bmatrix}
        \begin{array}{c}
            1 \\
            1 \\
            2 \\
            1
        \end{array}
    \end{bmatrix}
    +b \begin{bmatrix}
        \begin{array}{c}
            0  \\
            -2 \\
            1  \\
            0
        \end{array}
    \end{bmatrix}
    = 0
\end{math}

\singlespacing

is $a=0$ and $b=0$. Because you can't multiply the first vector by a scalar and
get the second vector.

\singlespacing

Now let's find the basis of $U_2$. We can notice that the first and second column
if we added them we would get the third column. So we can say that the third
column is a linear combination of the first and second column. Then for now we
can generate $U_2$ with the first and second column of $A_2$.

\singlespacing

But because the first and second column are linearly independent, we can say that
the basis of $U_2$ is the first and second column of $A_2$.

\singlespacing

The proof that the first and second column of $A_2$ are linearly independent is

\singlespacing

\begin{math}
    a \begin{bmatrix}
        \begin{array}{c}
            3 \\
            1 \\
            7 \\
            3
        \end{array}
    \end{bmatrix}
    +b \begin{bmatrix}
        \begin{array}{c}
            -3 \\
            2  \\
            -5 \\
            -1
        \end{array}
    \end{bmatrix}
    = 0
\end{math}

\singlespacing

is $a=0$ and $b=0$. Because you can't multiply the first vector by a scalar and
get the second vector.

\singlespacing

So we have that the basis of $U_1$ and $U_2$ are:

\singlespacing

\begin{math}
    B_1 = \left<\begin{bmatrix}
        \begin{array}{c}
            1 \\
            1 \\
            2 \\
            1
        \end{array}
    \end{bmatrix},
    \begin{bmatrix}
        \begin{array}{c}
            0  \\
            -2 \\
            1  \\
            0
        \end{array}
    \end{bmatrix}\right>,
    \quad
    B_2 = \left<\begin{bmatrix}
        \begin{array}{c}
            3 \\
            1 \\
            7 \\
            3
        \end{array}
    \end{bmatrix},
    \begin{bmatrix}
        \begin{array}{c}
            -3 \\
            2  \\
            -5 \\
            -1
        \end{array}
    \end{bmatrix}\right>
\end{math}

\singlespacing
\singlespacing

So the dimensions of $U_1$ and $U_2$ are:

\singlespacing

\begin{math}
    dim(U_1) = 2,\quad dim(U_2) = 2
\end{math}

\singlespacing
