\textbf{Problem 5:}

\singlespacing

\begin{math}
    A_1=\begin{bmatrix}
        \begin{array}{ccc}
            1 & 0  & 1  \\
            1 & -2 & -1 \\
            2 & 1  & 3  \\
            1 & 0  & 1
        \end{array}
    \end{bmatrix},\quad
    A_2=\begin{bmatrix}
        \begin{array}{ccc}
            3 & -3 & 0 \\
            1 & 2  & 3 \\
            7 & -5 & 2 \\
            3 & -1 & 2
        \end{array}
    \end{bmatrix}
\end{math}

\singlespacing
\singlespacing

Let's find the basis of $U_1$ and $U_2$ where where $U_1$ is spanned by the columns
of $A_1$ and $U_2$ is spanned by the columns of $A_2$.

\singlespacing

We can notice that in $A_1$ the sum of the first and second column is equal to the
third column, so we can say that the third column is a linear combination of the
first and second column. Then for now we can generate $U_1$ with the first and
second column of $A_1$.

\singlespacing

But because the first and second column are linearly independent, we can say that
the basis of $U_1$ is the first and second column of $A_1$.

\singlespacing

The proof that the first and second column of $A_1$ are linearly independent is
that the only solution to the equation:

\singlespacing

\begin{math}
    a \begin{bmatrix}
        \begin{array}{c}
            1 \\
            1 \\
            2 \\
            1
        \end{array}
    \end{bmatrix}
    +b \begin{bmatrix}
        \begin{array}{c}
            0  \\
            -2 \\
            1  \\
            0
        \end{array}
    \end{bmatrix}
    = 0
\end{math}

\singlespacing

is $a=0$ and $b=0$. Because you can't multiply the first vector by a scalar and
get the second vector.

\singlespacing

Now let's find the basis of $U_2$. We can notice that the first and second column
if we added them we would get the third column. So we can say that the third
column is a linear combination of the first and second column. Then for now we
can generate $U_2$ with the first and second column of $A_2$.

\singlespacing

But because the first and second column are linearly independent, we can say that
the basis of $U_2$ is the first and second column of $A_2$.

\singlespacing

The proof that the first and second column of $A_2$ are linearly independent is

\singlespacing

\begin{math}
    a \begin{bmatrix}
        \begin{array}{c}
            3 \\
            1 \\
            7 \\
            3
        \end{array}
    \end{bmatrix}
    +b \begin{bmatrix}
        \begin{array}{c}
            -3 \\
            2  \\
            -5 \\
            -1
        \end{array}
    \end{bmatrix}
    = 0
\end{math}

\singlespacing

is $a=0$ and $b=0$. Because you can't multiply the first vector by a scalar and
get the second vector.

\singlespacing

So we have that the basis of $U_1$ and $U_2$ are:

\singlespacing

\begin{math}
    B_1 = \left<\begin{bmatrix}
        \begin{array}{c}
            1 \\
            1 \\
            2 \\
            1
        \end{array}
    \end{bmatrix},
    \begin{bmatrix}
        \begin{array}{c}
            0  \\
            -2 \\
            1  \\
            0
        \end{array}
    \end{bmatrix}\right>,
    \quad
    B_2 = \left<\begin{bmatrix}
        \begin{array}{c}
            3 \\
            1 \\
            7 \\
            3
        \end{array}
    \end{bmatrix},
    \begin{bmatrix}
        \begin{array}{c}
            -3 \\
            2  \\
            -5 \\
            -1
        \end{array}
    \end{bmatrix}\right>
\end{math}

\singlespacing
\singlespacing

So the dimensions of $U_1$ and $U_2$ are:

\singlespacing

\begin{math}
    dim(U_1) = 2,\quad dim(U_2) = 2
\end{math}

\singlespacing
\singlespacing
\singlespacing

Let's find the basis of $U_1 \cap U_2$:

\singlespacing

The set of vectors that are in $U_1$ are:

\singlespacing

\begin{math}
    S=\left\{
    a\begin{bmatrix}
        \begin{array}{c}
            1 \\
            1 \\
            2 \\
            1
        \end{array}
    \end{bmatrix}+
    b\begin{bmatrix}
        \begin{array}{c}
            0  \\
            -2 \\
            1  \\
            0
        \end{array}
    \end{bmatrix},\quad a,b \in \mathbb{R}\right\}
\end{math}

\singlespacing

Then:

\singlespacing

\begin{math}
    S=\left\{
    \begin{bmatrix}
        \begin{array}{c}
            a    \\
            a-2b \\
            2a+b \\
            a
        \end{array}
    \end{bmatrix},\quad a,b \in \mathbb{R}\right\}
\end{math}

\singlespacing

The set of vectors that are in $U_2$ are:

\singlespacing

\begin{math}
    T=\left\{
    c\begin{bmatrix}
        \begin{array}{c}
            3 \\
            1 \\
            7 \\
            3
        \end{array}
    \end{bmatrix}+
    d\begin{bmatrix}
        \begin{array}{c}
            -3 \\
            2  \\
            -5 \\
            -1
        \end{array}
    \end{bmatrix},\quad c,d \in \mathbb{R}\right\}
\end{math}

\singlespacing

Then:

\singlespacing

\begin{math}
    T=\left\{
    \begin{bmatrix}
        \begin{array}{c}
            3c-3d \\
            c+2d  \\
            7c-5d \\
            3c-d
        \end{array}
    \end{bmatrix},\quad c,d \in \mathbb{R}\right\}
\end{math}

\singlespacing
\singlespacing

Now we will find the vectors in $S$ that are also in $T$:

\singlespacing

\begin{math}
    \begin{bmatrix}
        \begin{array}{c}
            a    \\
            a-2b \\
            2a+b \\
            a
        \end{array}
    \end{bmatrix}
    =
    \begin{bmatrix}
        \begin{array}{c}
            3c-3d \\
            c+2d  \\
            7c-5d \\
            3c-d
        \end{array}
    \end{bmatrix}
\end{math}

\singlespacing

Then:

\singlespacing

\begin{math}
    \begin{cases}
        \begin{array}{cc}
            a=3c-3d    & (\rom{1}) \\
            a-2b=c+2d  & (\rom{2}) \\
            2a+b=7c-5d & (\rom{3}) \\
            a=3c-d     & (\rom{4})
        \end{array}
    \end{cases}
\end{math}

\singlespacing

We get that:

\singlespacing

$a = 3c-3d$ and $a = 3c-d$, then $3c-3d = 3c-d$, then $-3d = -d$, then $d = 0$.

\singlespacing

Replacing $d$ in equations:

\singlespacing

\begin{math}
    \begin{cases}
        \begin{array}{cc}
            a=3c-3(0)    & (\rom{1}) \\
            a-2b=c+2(0)  & (\rom{2}) \\
            2a+b=7c-5(0) & (\rom{3}) \\
            a=3c-(0)     & (\rom{4})
        \end{array}
    \end{cases}
\end{math}

\singlespacing

\begin{math}
    \begin{cases}
        \begin{array}{cc}
            a=3c    & (\rom{1}) \\
            a-2b=c  & (\rom{2}) \\
            2a+b=7c & (\rom{3}) \\
            a=3c    & (\rom{4})
        \end{array}
    \end{cases}
\end{math}

\singlespacing

Here we get that $a=3c$, so replacing equations $(\rom{2})$ and $(\rom{3})$;

\singlespacing

\begin{math}
    \begin{cases}
        \begin{array}{cc}
            3c-2b=c & (\rom{2}) \\
            6c+b=7c & (\rom{3})
        \end{array}
    \end{cases}
\end{math}

\singlespacing

multiplying $(\rom{1})$ by $-2$ and adding it to $(\rom{2})$ we get:

\singlespacing

$3c-2b=c$

$6c+b=7c$

\singlespacing

$-6c+4b=-2c \quad\cdot(-2)$

$6c+b=7c$

\singlespacing

And adding them we get:

\singlespacing

$5b=5c$

\singlespacing

Then $b=c$, so we get that $a=3c$, $b=c$ and $d=0$.

\singlespacing

So because $c$ is a free variable, and $d=0$, substituting the following:

\singlespacing

\begin{math}
    T=\left\{
    c\begin{bmatrix}
        \begin{array}{c}
            3 \\
            1 \\
            7 \\
            3
        \end{array}
    \end{bmatrix}+
    d\begin{bmatrix}
        \begin{array}{c}
            -3 \\
            2  \\
            -5 \\
            -1
        \end{array}
    \end{bmatrix},\quad c,d \in \mathbb{R}\right\}
\end{math}


\singlespacing

\begin{math}
    c\begin{bmatrix}
        \begin{array}{c}
            3 \\
            1 \\
            7 \\
            3
        \end{array}
    \end{bmatrix}+
    d\begin{bmatrix}
        \begin{array}{c}
            -3 \\
            2  \\
            -5 \\
            -1
        \end{array}
    \end{bmatrix}=
    c\begin{bmatrix}
        \begin{array}{c}
            3 \\
            1 \\
            7 \\
            3
        \end{array}
    \end{bmatrix}+
    0\begin{bmatrix}
        \begin{array}{c}
            -3 \\
            2  \\
            -5 \\
            -1
        \end{array}
    \end{bmatrix}=
    c\begin{bmatrix}
        \begin{array}{c}
            3 \\
            1 \\
            7 \\
            3
        \end{array}
    \end{bmatrix}
\end{math}

\singlespacing
\singlespacing
\singlespacing

Then we have that $c$ is a free variable (scalar), so the basis of $U_1 \cap U_2$ is:

\singlespacing
\singlespacing

\begin{math}
    B_1 \cap B_2 = \left<\begin{bmatrix}
        \begin{array}{c}
            3 \\
            1 \\
            7 \\
            3
        \end{array}
    \end{bmatrix}\right>
\end{math}

