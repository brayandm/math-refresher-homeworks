\textbf{Problem 6:}

\singlespacing

\begin{math}
    \Phi: \mathbb{R}^3 \rightarrow \mathbb{R}^4
\end{math}

\singlespacing
\singlespacing

\begin{math}
    \Phi\left(\begin{bmatrix}
        \begin{array}{c}
            x_1 \\
            x_2 \\
            x_3
        \end{array}
    \end{bmatrix}
    \right)=\begin{bmatrix}
        \begin{array}{c}
            3x_1 + 2x_2 + x_3 \\
            x_1 + x_2 + x_3   \\
            x_1 - 3x_2        \\
            2x_1 + 3x_2 + x_3
        \end{array}
    \end{bmatrix}
\end{math}

\singlespacing
\singlespacing

Let's find the transformation matrix $A_\Phi$. We have that:

\singlespacing

\begin{math}
    \Phi(\vec{x})=A_\Phi \times \vec{x}
\end{math}

\singlespacing

So The transformation matrix is defined as:

\singlespacing

\begin{math}
    \begin{bmatrix}
        \begin{array}{c}
            3x_1 + 2x_2 + x_3 \\
            x_1 + x_2 + x_3   \\
            x_1 - 3x_2        \\
            2x_1 + 3x_2 + x_3
        \end{array}
    \end{bmatrix} =A_\Phi \times\begin{bmatrix}
        \begin{array}{c}
            x_1 \\
            x_2 \\
            x_3
        \end{array}
    \end{bmatrix}
\end{math}

\singlespacing

So if we take the following obvious matrix:

\singlespacing

\begin{math}
    A_\Phi=\begin{bmatrix}
        \begin{array}{ccc}
            3 & 2   & 1 \\
            1 & 1   & 1 \\
            1 & - 3 & 0 \\
            2 & 3   & 1
        \end{array}
    \end{bmatrix}
\end{math}

\singlespacing

Then we can write:

\singlespacing

\begin{math}
    \begin{bmatrix}
        \begin{array}{c}
            3x_1 + 2x_2 + x_3 \\
            x_1 + x_2 + x_3   \\
            x_1 - 3x_2        \\
            2x_1 + 3x_2 + x_3
        \end{array}
    \end{bmatrix} = \begin{bmatrix}
        \begin{array}{ccc}
            3 & 2   & 1 \\
            1 & 1   & 1 \\
            1 & - 3 & 0 \\
            2 & 3   & 1
        \end{array}
    \end{bmatrix} \times\begin{bmatrix}
        \begin{array}{c}
            x_1 \\
            x_2 \\
            x_3
        \end{array}
    \end{bmatrix}
\end{math}

\singlespacing

Then the equation is satisfied and we found the transformation matrix.

\singlespacing
\singlespacing

Now let's find the $\text{rk}(A_\Phi)$:

\singlespacing

We have that:

\singlespacing

\begin{math}
    \text{rk}(A_\Phi)=\text{rk}\left(\begin{bmatrix}
            \begin{array}{ccc}
                3 & 2   & 1 \\
                1 & 1   & 1 \\
                1 & - 3 & 0 \\
                2 & 3   & 1
            \end{array}
        \end{bmatrix}\right)
\end{math}

\singlespacing

Let's apply Gauss elimination to find the rank of the matrix by counting the number of not null rows:

\singlespacing

\begin{math}
    \begin{bmatrix}
        \begin{array}{ccc}
            3 & 2  & 1 \\
            1 & 1  & 1 \\
            1 & -3 & 0 \\
            2 & 3  & 1
        \end{array}
    \end{bmatrix} \xrightarrow{R_4 = R_4 + (-2R_2)}
    \begin{bmatrix}
        \begin{array}{ccc}
            3 & 2  & 1  \\
            1 & 1  & 1  \\
            1 & -3 & 0  \\
            0 & 1  & -1
        \end{array}
    \end{bmatrix} \xrightarrow{R_3 = R_3 + (-R_2)}
    \begin{bmatrix}
        \begin{array}{ccc}
            3 & 2  & 1  \\
            1 & 1  & 1  \\
            0 & -4 & -1 \\
            0 & 1  & -1
        \end{array}
    \end{bmatrix} \xrightarrow{R_1 = R_1 + (-3R_2)}
\end{math}

\singlespacing

\begin{math}
    \begin{bmatrix}
        \begin{array}{ccc}
            0 & -1 & -2 \\
            1 & 1  & 1  \\
            0 & -4 & -1 \\
            0 & 1  & -1
        \end{array}
    \end{bmatrix} \xrightarrow{swap(R_1,R_2)}
    \begin{bmatrix}
        \begin{array}{ccc}
            1 & 1  & 1  \\
            0 & -1 & -2 \\
            0 & -4 & -1 \\
            0 & 1  & -1
        \end{array}
    \end{bmatrix}\xrightarrow{R_3 = R_3 + (-4R_2)}
    \begin{bmatrix}
        \begin{array}{ccc}
            1 & 1  & 1  \\
            0 & -1 & -2 \\
            0 & 0  & 7  \\
            0 & 1  & -1
        \end{array}
    \end{bmatrix}\xrightarrow{R_4 = R_4 + R_2}
\end{math}

\begin{math}
    \begin{bmatrix}
        \begin{array}{ccc}
            1 & 1  & 1  \\
            0 & -1 & -2 \\
            0 & 0  & 7  \\
            0 & 0  & -3
        \end{array}
    \end{bmatrix} \xrightarrow{R_4 = R_4 + \frac{3}{7}R_3}
    \begin{bmatrix}
        \begin{array}{ccc}
            1 & 1  & 1  \\
            0 & -1 & -2 \\
            0 & 0  & 7  \\
            0 & 0  & 0
        \end{array}
    \end{bmatrix}
\end{math}

\singlespacing
\singlespacing

So we have that $\text{rk}(A_\Phi)=3$, because there are 3 not null rows.

\singlespacing

Let's compute the kernel of $\Phi$ and its dimension:

\singlespacing

The kernel of $\Phi$ is defined as:

\singlespacing

\begin{math}
    \ker(\Phi)=\left\{\vec{x} \in \mathbb{R}^3 \mid \Phi(\vec{x})=\vec{0}\right\}
\end{math}

\singlespacing

So we have that:

\singlespacing

\begin{math}
    \Phi(\vec{x})=\vec{0} \Rightarrow \begin{bmatrix}
        \begin{array}{c}
            3x_1 + 2x_2 + x_3 \\
            x_1 + x_2 + x_3   \\
            x_1 - 3x_2        \\
            2x_1 + 3x_2 + x_3
        \end{array}
    \end{bmatrix} = \begin{bmatrix}
        \begin{array}{c}
            0 \\
            0 \\
            0 \\
            0
        \end{array}
    \end{bmatrix}
\end{math}

\singlespacing

So we have that:

\singlespacing

\begin{math}
    \begin{cases}
        3x_1 + 2x_2 + x_3 = 0 \\
        x_1 + x_2 + x_3 = 0   \\
        x_1 - 3x_2 = 0        \\
        2x_1 + 3x_2 + x_3 = 0
    \end{cases}
\end{math}

\singlespacing

Then try to solve the system using Gauss elimination:

\singlespacing

\begin{math}
    \begin{bmatrix}
        \begin{array}{ccc|c}
            3 & 2  & 1 & 0 \\
            1 & 1  & 1 & 0 \\
            1 & -3 & 0 & 0 \\
            2 & 3  & 1 & 0
        \end{array}
    \end{bmatrix}
\end{math}

\singlespacing

But we already applied Gauss elimination to find the rank of the matrix, so we can use the matrix we already have:

\singlespacing

\begin{math}
    \begin{bmatrix}
        \begin{array}{ccc|c}
            1 & 1  & 1  & 0 \\
            0 & -1 & -2 & 0 \\
            0 & 0  & 7  & 0 \\
            0 & 0  & 0  & 0
        \end{array}
    \end{bmatrix}
\end{math}

\singlespacing

So we have that:

\singlespacing

\begin{math}
    \begin{cases}
        x_1 + x_2 + x_3 = 0 \\
        -x_2 - 2x_3 = 0     \\
        7x_3 = 0
    \end{cases}
\end{math}

\singlespacing

Finding the solution of the system we have that:

\singlespacing

$7x_3 = 0$

$x_3 = 0$

\singlespacing

$-x_2 - 2x_3 = 0$

$-x_2 - 2(0) = 0$

$-x_2 = 0$

$x_2 = 0$

\singlespacing

$x_1 + x_2 + x_3 = 0$

$x_1 + 0 + 0 = 0$

$x_1 = 0$

\singlespacing

So we have that the solution of the system is the following vector:

\singlespacing

\begin{math}
    \vec{x}=\begin{bmatrix}
        \begin{array}{c}
            0 \\
            0 \\
            0
        \end{array}
    \end{bmatrix}
\end{math}

\singlespacing

Then $\text{ker}(\Phi) =  \{\mathbf{0}\}$ and $\text{dim}(\text{ker}(\Phi)) = 0$.

\singlespacing

Let's compute the image of $\Phi$ and its dimension:

\singlespacing

The image of $\Phi$ is defined as:

\singlespacing

\begin{math}
    \text{Im}(\Phi)=\left\{\vec{y} \in \mathbb{R}^4 \mid \exists \vec{x} \in \mathbb{R}^3 \text{ such that } \Phi(\vec{x})=\vec{y}\right\}
\end{math}

\singlespacing

Then we can extract easily the image of $\Phi$ from the transformation matrix $A_\Phi$:

\singlespacing

\begin{math}
    \text{Im}(\Phi)=\left\{
    \begin{bmatrix}
        \begin{array}{c}
            3x_1 + 2x_2 + x_3 \\
            x_1 + x_2 + x_3   \\
            x_1 - 3x_2        \\
            2x_1 + 3x_2 + x_3
        \end{array}
    \end{bmatrix}
    \mid x_1, x_2, x_3 \in \mathbb{R}
    \right\}
\end{math}

\singlespacing

Then we have the definition that says that the dimension of the image of a transformation is the same as the rank of the transformation matrix, so we have that $\text{dim} (\text{Im}(\Phi)) = 3$.