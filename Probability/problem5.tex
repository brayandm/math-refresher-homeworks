\textbf{Problem 5:}

\singlespacing

\textbf{Solution:}

\singlespacing

\textbf{Part 1)}

\singlespacing

To find the distribution function $F$ of $P$ we need to integrate the density function $p$:

\singlespacing

\begin{equation}
    F(t) = \int_{-\infty}^{t} p(u) \, du
\end{equation}

\singlespacing

We have that $p(t) = \frac{I_{[a,b]}(t)}{b-a}$, so:

\singlespacing

\begin{equation}
    F(t) = \int_{-\infty}^{t} \frac{I_{[a,b]}(u)}{b-a} \, du
\end{equation}

\singlespacing

We have 3 cases:

\singlespacing

\begin{enumerate}
    \item $t < a$
    \item $a \leq t \leq b$
    \item $t > b$
\end{enumerate}

\singlespacing

\textbf{Case 1:} $t < a$

\singlespacing

In this case $I_{[a,b]}(u) = 0$, so:

\singlespacing

\begin{equation}
    F(t) = \int_{-\infty}^{t} \frac{I_{[a,b]}(u)}{b-a} \, du = \int_{-\infty}^{t} 0 \, du = 0
\end{equation}

\singlespacing

\textbf{Case 2:} $a \leq t \leq b$

\singlespacing

In this case $I_{[a,b]}(u) = 1$, and we only need to integrate from $a$ to $t$,
because the density function is zero outside of $[a, b]$:

\singlespacing

\begin{equation}
    F(t) = \int_{-\infty}^{t} \frac{I_{[a,b]}(u)}{b-a} \, du = \int_{a}^{t} \frac{1}{b-a} \, du = \Big(\frac{u}{b-a} \Big)\Big|_{a}^{t} = \frac{t}{b-a} - \frac{a}{b-a} = \frac{t - a}{b-a}
\end{equation}

\singlespacing

\textbf{Case 3:} $t > b$

\singlespacing

In this case $I_{[a,b]}(u) = 1$, and we need to integrate from $a$ to $b$,
because the density function is zero outside of $[a, b]$:

\singlespacing

\begin{equation}
    F(t) = \int_{-\infty}^{t} \frac{I_{[a,b]}(u)}{b-a} \, du = \int_{a}^{b} \frac{1}{b-a} \, du = \Big(\frac{u}{b-a} \Big)\Big|_{a}^{b} = \frac{b}{b-a} - \frac{a}{b-a} = \frac{b - a}{b-a} = 1
\end{equation}

\singlespacing

So, the distribution function $F$ of $P$ is:

\singlespacing

\begin{equation}
    F(t) = \begin{cases}
        0                 & \text{if } t < a           \\
        \frac{t - a}{b-a} & \text{if } a \leq t \leq b \\
        1                 & \text{if } t > b
    \end{cases}
\end{equation}

\singlespacing

\textbf{Part 2)}

\singlespacing

To find the distribution function $F$ of $P$ we need to integrate the density function $p$:

\singlespacing

\begin{equation}
    F(t) = \int_{-\infty}^{t} p(u) \, du
\end{equation}

\singlespacing

We have that $p(t) = \lambda e^{-\lambda t}I_{(0,\infty)}(t)$, so:

\singlespacing

\begin{equation}
    F(t) = \int_{-\infty}^{t} \lambda e^{-\lambda u}I_{(0,\infty)}(u) \, du
\end{equation}

\singlespacing

We have 2 cases:

\singlespacing

\begin{enumerate}
    \item $t < 0$
    \item $t \geq 0$
\end{enumerate}

\singlespacing

\textbf{Case 1:} $t < 0$

\singlespacing

In this case $I_{(0,\infty)}(u) = 0$, so:

\singlespacing

\begin{equation}
    F(t) = \int_{-\infty}^{t} \lambda e^{-\lambda u}I_{(0,\infty)}(u) \, du = \int_{-\infty}^{t} 0 \, du = 0
\end{equation}

\singlespacing

\textbf{Case 2:} $t \geq 0$

\singlespacing

In this case $I_{(0,\infty)}(u) = 1$, and we only need to integrate from $0$ to $t$,

\singlespacing

\begin{equation}
    F(t) = \int_{-\infty}^{t} \lambda e^{-\lambda u}I_{(0,\infty)}(u) \, du = \int_{0}^{t} \lambda e^{-\lambda u} \, du = \Big(-e^{-\lambda u} \Big)\Big|_{0}^{t} = -e^{-\lambda t} + e^{-\lambda 0} = 1 - e^{-\lambda t}
\end{equation}

\singlespacing

So, the distribution function $F$ of $P$ is:

\singlespacing

\begin{equation}
    F(t) = \begin{cases}
        0                  & \text{if } t < 0    \\
        1 - e^{-\lambda t} & \text{if } t \geq 0
    \end{cases}
\end{equation}

\singlespacing

\textbf{Part 3)}

\singlespacing

We have that:

\singlespacing

\begin{equation}
    F_1(t) = \begin{cases}
        0                 & \text{if } t < a           \\
        \frac{t - a}{b-a} & \text{if } a \leq t \leq b \\
        1                 & \text{if } t > b
    \end{cases}
\end{equation}

\singlespacing

\begin{equation}
    F_2(t) = \begin{cases}
        0                  & \text{if } t < 0    \\
        1 - e^{-\lambda t} & \text{if } t \geq 0
    \end{cases}
\end{equation}

\singlespacing

We need to find $F_1(2b)F_2(\frac{\text{ln}(2)}{\lambda})$:

\singlespacing
\singlespacing

We have three cases for $F_1(2b)$:

\singlespacing

\begin{enumerate}
    \item $2b < a$
    \item $a \leq 2b \leq b$
    \item $2b > b$
\end{enumerate}

\singlespacing

\textbf{Case 1:} $2b < a$

\singlespacing

In this case $F_1(2b) = 0$

\singlespacing

\textbf{Case 2:} $a \leq 2b \leq b$

\singlespacing

In this case $F_1(2b) = \frac{2b - a}{b-a}$

\singlespacing

\textbf{Case 3:} $2b > b$

\singlespacing

In this case $F_1(2b) = 1$

\singlespacing
\singlespacing

We have three cases for $F_2(\frac{\text{ln}(2)}{\lambda})$:

\singlespacing

\begin{enumerate}
    \item $\lambda < 0$
    \item $\lambda = 0$
    \item $\lambda > 0$
\end{enumerate}

\singlespacing

\textbf{Case 1:} $\lambda < 0$

\singlespacing

In this case $F_2(\frac{\text{ln}(2)}{\lambda}) = 0$, because $\frac{\text{ln}(2)}{\lambda} < 0$

\singlespacing

\textbf{Case 2:} $\lambda = 0$

\singlespacing

In this case $F_2(\frac{\text{ln}(2)}{\lambda})$ is \textbf{undefined}, because $\frac{\text{ln}(2)}{\lambda} = \frac{\text{ln}(2)}{0}$ is \textbf{undefined}

\singlespacing

\textbf{Case 3:} $\lambda > 0$

\singlespacing

In this case $F_2(\frac{\text{ln}(2)}{\lambda}) = 1 - e^{-\lambda \frac{\text{ln}(2)}{\lambda}} = 1 - e^{-\text{ln}(2)} = 1 - \frac{1}{2} = \frac{1}{2}$

\singlespacing

So, $F_1(2b)F_2(\frac{\text{ln}(2)}{\lambda})$ is:

\singlespacing

\begin{equation}
    F_1(2b)F_2(\frac{\text{ln}(2)}{\lambda}) = \begin{cases}
        undefined                            & \lambda = 0                                           \\
        0                                    & \text{if } 2b < a \text{ or } \lambda < 0 \text{ or } \\
        \frac{2b - a}{b-a}\times \frac{1}{2} & \text{if } a \leq 2b \leq b \text{ and } \lambda > 0  \\
        1\times \frac{1}{2}                  & \text{if } 2b > b \text{ and } \lambda > 0
    \end{cases}
\end{equation}