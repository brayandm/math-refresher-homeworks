\textbf{Problem 2:}

\singlespacing

Let's calculate the probability that the value 1 never appears. The probability
of that is given by the number of sequences of $n$ rolls that don't have the value 1,
that is $5^n$, divided by the total number of sequences of $n$ rolls, that is $6^n$.
Then the probability that the value 1 never appears is given by:

\singlespacing

\begin{equation}
    \frac{5^n}{6^n}
\end{equation}

\singlespacing

So we have 6 possible values, then the probability that at least 1 of the 6 values
never appears is given by:

\singlespacing

\begin{equation}
    6 \cdot\frac{5^n}{6^n}
\end{equation}

\singlespacing

But let's notice that we are counting more than once the sequences that don't have
two values, for example, we are counting twice the sequences that don't have the
values 1 and 2, because we are counting them when we calculate the probability that
the value 1 never appears and when we calculate the probability that the value 2 never
appears. Then we need to subtract the sequences that don't have two values, that is
$4^n$, and because we have $\binom{6}{2}$ ways to choose two values, then the probability
that at least 1 of the 6 values never appears is given by:

\singlespacing

\begin{equation}
    6 \cdot\frac{5^n}{6^n} - \binom{6}{2} \cdot \frac{4^n}{6^n}
\end{equation}

\singlespacing

But then we are subtracting too much, because we are subtracting the sequences that
don't have three values three times, then we need to add them back.

\singlespacing

\begin{equation}
    6 \cdot\frac{5^n}{6^n} - \binom{6}{2} \cdot \frac{4^n}{6^n} + \binom{6}{3} \cdot \frac{3^n}{6^n}
\end{equation}

\singlespacing

So seeing this pattern, we can notice that this is given by the inclusion-exclusion principle,
then the probability that at least 1 of the 6 values never appears is given by:

\singlespacing

\begin{equation}
    \sum_{i=1}^{5} (-1)^{i+1} \binom{6}{i} \frac{(6 - i)^n}{6^n}
\end{equation}

\singlespacing

*Note that $\binom{n}{r} = \frac{n!}{r! (n - r)!}$