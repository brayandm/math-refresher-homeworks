\textbf{Problem 2:}

\singlespacing

Instead of calculating the probability of the event of at least 1 of the 6 values never appearing,
we can calculate the probability of the event of exactly x values never appearing for $x \in \{1, 2, 3, 4, 5\}$,
and then sum the probabilities of those events.

\singlespacing

For calculating the probability of the event of exactly x values never appearing
first we need to select x values from the 6 possible values to be the missing values,
and then we need to calculate the probability of each sequence of n rolls having exactly
those x values missing.

\singlespacing

The number of ways of selecting x values from 6 possible values is given by the binomial coefficient $\binom{6}{x}$.

\singlespacing

So because we have 6 possible values and we want to choose x of them to be missing we will
have $6 - x$ possible values that will appear in the sequence of n rolls,
and then the total of these sequences will be $(6 - x)^n$. But because there are $\binom{6}{x}$
possible ways of choosing the missing values, then the total of sequences with exactly x values
missing will be $\binom{6}{x} (6 - x)^n$.

\singlespacing

Now to get the probability of each sequence we need to divide the total of sequences with exactly x values
missing by the total of possible sequences, that is $6^n$.

\singlespacing

Then the probability of the event of exactly x values never appearing is given by:

\singlespacing

\begin{equation}
    \binom{6}{x} \frac{(6 - x)^n}{6^n}
\end{equation}

\singlespacing

Now we need to sum the probabilities of the events of exactly x values never appearing for $x \in \{1, 2, 3, 4, 5\}$,

\singlespacing

\begin{equation}
    \sum_{x=1}^{5} \binom{6}{x} \frac{(6 - x)^n}{6^n}
\end{equation}