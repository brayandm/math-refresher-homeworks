\textbf{Problem 6:}

\singlespacing

By definition of expectation we have that:

\singlespacing

\begin{equation}
    \mathbb{E}[X] = \sum_{k: P(X = k) > 0} k \times \mathbb{P}(X = k)
\end{equation}

\singlespacing

So, because X has Poisson distribution, we have that:

\singlespacing

\begin{equation}
    \mathbb{E}[X] = \sum_{k=0}^{\infty} k \frac{\lambda^k}{k!}e^{-\lambda}
\end{equation}

\singlespacing

Let's extract $e^{-\lambda}$ from the sum, because it is a constant:

\singlespacing

\begin{equation}
    \mathbb{E}[X] = e^{-\lambda} \sum_{k=0}^{\infty} k \frac{\lambda^k}{k!}
\end{equation}

\singlespacing

We can notice that when $k = 0$ the term $k \frac{\lambda^k}{k!}$ is equal to $0$,
so we can start the sum from $k = 1$:

\singlespacing

\begin{equation}
    \mathbb{E}[X] = e^{-\lambda} \sum_{k=1}^{\infty} k \frac{\lambda^k}{k!}
\end{equation}

\singlespacing

\begin{equation}
    \mathbb{E}[X] = e^{-\lambda} \sum_{k=1}^{\infty} \frac{\lambda^k}{(k-1)!}
\end{equation}

\singlespacing

Let's extract now one $\lambda$ from the sum, because it is a constant:

\singlespacing

\begin{equation}
    \mathbb{E}[X] = e^{-\lambda} \lambda \sum_{k=1}^{\infty} \frac{\lambda^{k-1}}{(k-1)!}
\end{equation}

\singlespacing

Now we can see that the sum seems to be the Taylor series of $e^{\lambda}$, but
starting from $k = 1$ instead of $k = 0$. Then if we substitute $k - 1$ with $t$:

\singlespacing

\begin{equation}
    \mathbb{E}[X] = e^{-\lambda} \lambda \sum_{t=0}^{\infty} \frac{\lambda^{t}}{t!}
\end{equation}

\singlespacing

\break

Then the sum is equal to $e^{\lambda}$:

\singlespacing

\begin{equation}
    \mathbb{E}[X] = e^{-\lambda} \lambda e^{\lambda}
\end{equation}

\singlespacing

Multiplying the two exponentials we get:

\singlespacing

\begin{equation}
    \mathbb{E}[X] = \lambda
\end{equation}

\singlespacing

So the expectation of $X$ is $\lambda$.