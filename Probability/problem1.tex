\textbf{Problem 1:}

\singlespacing

In this problem because we are extracting balls with replacement
the probability of extracting a red ball is the same for each extraction, also
the action of extracting a ball is independent of the previous extractions.

\singlespacing

Let's define the probability of extracting a red ball as $P(A) = \frac{m_1}{m}$, where
$A$ is the event of extracting a red ball, and the probability of extracting a green ball as $P(B) = \frac{m_2}{m}$, where
$B$ is the event of extracting a green ball.

\singlespacing

In each extraction we can have two possible outcomes, either we extract a red ball
or we extract a green ball. Then in $n$ extractions we can have $2^n$ possible outcomes for
the total sequence of extractions. But it is easy to calculate that the number of sequences
with exactly $r$ red balls is given by the binomial coefficient $\binom{n}{r}$, because
we have $n$ extractions and we want to choose $r$ of them to be red.

\singlespacing

Then because the extractions are independent, if we have a sequence of $n$ extractions
with exactly $r$ red balls the probability of that sequence is given by the product of the
probabilities of each extraction, that is $P(A)^r P(B)^{n-r}$.

\singlespacing

Then the probability of having a sequence of $n$ extractions with exactly $r$ red balls is given by:

\singlespacing

\begin{equation}
    \binom{n}{r} P(A)^r P(B)^{n-r}
\end{equation}

\singlespacing

That's the amount of sequences with exactly $r$ red balls multiplied by the probability of each sequence.