\textbf{Problem 4:}

\singlespacing

Let $P$ be a distribution on $\mathbb{R}$ and $F$ be its distribution function. Are
the following equalities true? $F(a-)$ denotes one-sided limit where function argument
approaches a from the left.

\begin{enumerate}
    \item $P([a, b]) = F(b) - F(a -)$
    \item $P((a, b)) = F(b) - F(a)$
    \item $P([a, b)) = F(b -) - F(a -)$
    \item $P(\{x\}) = F(x) - F(x -)$
\end{enumerate}


\textbf{1)}

\singlespacing

We have that $P([a, b])$ is the probability that $X$ is in the interval $[a, b]$, then:

\singlespacing

\begin{equation}
    P([a, b]) = P(a \leq X \leq b) = P(X \leq b) - P(X < a)
\end{equation}

\singlespacing

This is because we include all the values of $X$ that are less than or equal to $b$,
but we exclude all the values of $X$ that are less than $a$.

\singlespacing

Then because $F$ is the distribution function of $P$ we have that:

\singlespacing

\begin{equation}
    P(X \leq b) - P(X < a) = F(b) - F(a -)
\end{equation}

\singlespacing

So $P([a, b]) = F(b) - F(a -)$ and the equality is \textbf{true}.

\singlespacing
\singlespacing
\singlespacing

\textbf{2)}

\singlespacing

We have that $P((a, b))$ is the probability that $X$ is in the interval $(a, b)$, then:

\singlespacing

\begin{equation}
    P((a, b)) = P(a < X < b) = P(X < b) - P(X \leq a)
\end{equation}

\singlespacing

This is because we include all the values of $X$ that are less than $b$,
but we exclude all the values of $X$ that are less than or equal to $a$.

\singlespacing

Then because $F$ is the distribution function of $P$ we have that:

\singlespacing

\begin{equation}
    P(X < b) - P(X \leq a) = F(b-) - F(a)
\end{equation}

\singlespacing

So $P((a, b)) = F(b-) - F(a)$ and the equality is \textbf{false}.

\singlespacing

\break

\textbf{3)}

\singlespacing

We have that $P([a, b))$ is the probability that $X$ is in the interval $[a, b)$, then:

\singlespacing

\begin{equation}
    P([a, b)) = P(a \leq X < b) = P(X < b) - P(X < a)
\end{equation}

\singlespacing

This is because we include all the values of $X$ that are less than $b$,
but we exclude all the values of $X$ that are less than $a$.

\singlespacing

Then because $F$ is the distribution function of $P$ we have that:

\singlespacing

\begin{equation}
    P(X < b) - P(X < a) = F(b-) - F(a-)
\end{equation}

\singlespacing

So $P([a, b)) = F(b-) - F(a-)$ and the equality is \textbf{true}.

\singlespacing
\singlespacing
\singlespacing

\textbf{4)}

\singlespacing

We have that $P(\{x\})$ is the probability that $X$ takes the value $x$, then:

\singlespacing

\begin{equation}
    P(\{x\}) = P(X = x) = P(X \leq x) - P(X < x)
\end{equation}

\singlespacing

This is because we include all the values of $X$ that are less than or equal to $x$,
but we exclude all the values of $X$ that are less than $x$.

\singlespacing

Then because $F$ is the distribution function of $P$ we have that:

\singlespacing

\begin{equation}
    P(X \leq x) - P(X < x) = F(x) - F(x-)
\end{equation}

\singlespacing

So $P(\{x\}) = F(x) - F(x-)$ and the equality is \textbf{true}.