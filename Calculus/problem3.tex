\textbf{Problem 3:}

\singlespacing

$f(x) = x^TAc,\quad A \in \mathbb{R}^{N\times N},\quad x \in \mathbb{R}^N, \quad c \in \mathbb{R}^N$

\singlespacing

Then we have to calculate $\frac{\partial f(x)}{\partial x}$

\singlespacing

Because $f(x)$ is a scalar and $x$ is a vector then $\frac{\partial f(x)}{\partial x}$
can be written as the following vector:

\singlespacing

{
    \renewcommand{\arraystretch}{1.8}
    $\frac{\partial f(x)}{\partial x} = \begin{bmatrix}
            \frac{\partial f(x)}{\partial x_1} \\
            \frac{\partial f(x)}{\partial x_2} \\
            \vdots                             \\
            \frac{\partial f(x)}{\partial x_N}
        \end{bmatrix}$
}

\singlespacing

Now we will calculate each partial derivative. But before
let's define vector $b$ as $b = Ac$, because matrix $A$ multiplied
by vector $c$ will result in a vector $b$ with $N$ elements.

\singlespacing

So we have now $f(x) = x^Tb$

\singlespacing

Then $\frac{\partial f(x)}{\partial x_i} = \frac{\partial x^Tb}{\partial x_i} = \frac{\partial \sum_{i=1}^{N}x_i\cdot b_i}{\partial x_i}$

\singlespacing

Because all $x_j$ with $j \neq i$ are constants and when
$i = j$ then $\frac{\partial x_i\cdot b_i}{\partial x_i} = b_i$

\singlespacing

Repeating this for all $i$ we get:

\singlespacing

{
    \renewcommand{\arraystretch}{1.8}
    $\frac{\partial f(x)}{\partial x} = \begin{bmatrix}
            b_1    \\
            b_2    \\
            \vdots \\
            b_N
        \end{bmatrix}$
}

\singlespacing

So we get at the end that:

\singlespacing


$\frac{\partial f(x)}{\partial x} = b$

\singlespacing

Then replacing $b$ with $Ac$ we get:

\singlespacing

$\frac{\partial f(x)}{\partial x} = Ac$

