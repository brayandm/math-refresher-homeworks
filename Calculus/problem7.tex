\textbf{Problem 7:}

\singlespacing


\begin{equation}
    \sum_{n=1}^{\infty} \frac{1}{n^p}
\end{equation}

\singlespacing

First let's notice that if $p<=0$ then $\frac{1}{n^p} <= \frac{1}{(n+1)^p}$,
so the series is increasing, also we can notice that the series is positive.

\singlespacing

Then taking $n=1$ we have $\frac{1}{1^p} = 1$ for any $p$ and the rest of the next terms are
greater or equal than $1$, so the sum of infinite terms greater or equal than $1$ is infinite,

\singlespacing

So the series diverges for $p<=0$.

\singlespacing
\singlespacing
\singlespacing

Now let's take $p>0$

\singlespacing

We have that if $p>0$ then $\frac{1}{n^p} > \frac{1}{(n+1)^p}$, so the series is decreasing. Also we can notice that $\frac{1}{n^p} > 0$.

\singlespacing

Then the series is positive, monotone decreasing and it is defined in interval $[1, \infty)$

\singlespacing

So we can apply the integral test for convergence:

\singlespacing

A infinite series $\sum_{n=N}^{\infty} f(n)$ converges if and only if the improper integral $\int_{N}^{\infty} f(x) dx$ converges.

\singlespacing

So taking $f(x) = \frac{1}{x^p} = x^{-p}$

\singlespacing

We have that:

\singlespacing

$\int_{1}^{\infty} x^{-p} dx = \lim_{b \to \infty} \int_{1}^{b} x^{-p} dx$

\singlespacing

Then the integral of $x^{-p}$ is $\frac{x^{-p+1}}{-p+1}$ by the power rule.

\singlespacing

$\lim_{b \to \infty} \int_{1}^{b} x^{-p} dx = \lim_{b \to \infty} \frac{x^{-p+1}}{-p+1} \Big|_{1}^{b} = \lim_{b \to \infty} \frac{b^{-p+1}}{-p+1} - \frac{1^{-p+1}}{-p+1} = \lim_{b \to \infty} \frac{b^{-p+1}}{-p+1} + \frac{1}{p-1}$

\singlespacing

Now let's analyze three cases:

\singlespacing

\textbf{a)} When $0<p<1$

\singlespacing

Then $-p+1 > 0$ and $\lim_{b \to \infty} \frac{b^{-p+1}}{-p+1}  + \frac{1}{p-1} = \lim_{b \to \infty} \frac{b^{-p+1}}{-p+1} = \lim_{b \to \infty} b^{-p+1} = \infty$

\singlespacing

Because $b \to \infty$ and $-p+1 > 0$ so $b^{-p+1} \to \infty$ and the series diverges.

\singlespacing

\textbf{b)} When $p=1$

\singlespacing

In this case $f(x) = 1/x$ so $\int f(x) dx = \ln(x)$

\singlespacing

Then $\lim_{b \to \infty} \int_{1}^{b} \frac{1}{x} dx = \lim_{b \to \infty} \ln(b) - \ln(1) = \lim_{b \to \infty} \ln(b) = \infty$

\singlespacing

So the series diverges.

\singlespacing

\textbf{c)} When $p>1$

\singlespacing

Then $-p+1 < 0$ and $\lim_{b \to \infty} \frac{b^{-p+1}}{-p+1}  + \frac{1}{p-1} = \lim_{b \to \infty} \frac{b^{-p+1}}{-p+1} = \lim_{b \to \infty} b^{-p+1} = 0$

\singlespacing

Because $b \to \infty$ and $-p+1 < 0$ so $b^{-p+1} \to 0$ and the series converges.

\singlespacing
\singlespacing
\singlespacing

So the series converges for $p>1$ and diverges for $p<=1$.